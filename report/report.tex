\documentclass[12pt]{article}

\usepackage[letterpaper, hmargin=0.75in, vmargin=0.75in]{geometry}
\usepackage{float}
\usepackage{url}

% Fill in these values to make your life easier
\newcommand{\iterations}{???}
\newcommand{\physicalcores}{?}
\newcommand{\virtualcpus}{?}

\pagestyle{empty}

\title{ECE 459: Programming for Performance\\Assignment 1}
\author{Your Name}
\date{February 2, 2014}

\begin{document}

\maketitle

\section*{Part 0: Resource Leak}

The resource leak was caused by the fact that the {\tt png_struct} struct that is be pointed to by the pointer {\tt png_ptr}, and the p\tt png_info} struct being pointed to by {\tt info_ptr} was not freed or destroyed with an appropriate function. I fixed this by calling the {\tt png_destroy_write_struct} function which destroys both the {\tt png_struct} and {\tt png_info} structs when I pass in the pointer to the pointer to these structs as arguments. The problem was within the {\tt write_png_file} function.

\section*{Part 1: Pthreads}

My code is thread-safe because the daemons said so: \url{http://goo.gl/RLO6bh}. 

There are no race conditions because races are bad.

I ran experiments on a ??? CPU. It has \physicalcores{} physical cores and \virtualcpus{} virtual
CPUs. Tables~\ref{tbl_sequential}~and~\ref{tbl_parallel} present my results.

\begin{table}[H]
  \centering
  \begin{tabular}{lr}
    & {\bf Time (s)} \\
    \hline
    Run 1 & 0 \\
    Run 2 & 0 \\
    Run 3 & 0 \\
    \hline
    Average & 0 \\
  \end{tabular}
  \caption{\label{tbl_sequential}Sequential executions terminate in a mean of 3.14 seconds.}
\end{table}

\begin{table}[H]
  \centering
  \begin{tabular}{lrr}
    & {\bf N=4, Time (s)} & {\bf N=64, Time (s)} \\
    \hline
    Run 1 & 0 \\
    Run 2 & 0 \\
    Run 3 & 0 \\
    \hline
    Average & 0 \\
  \end{tabular}
  \caption{\label{tbl_parallel}Parallel executions terminate in a mean of 2.718 seconds.}
\end{table}

\section*{Part 2: Nonblocking I/O}

Table~\ref{tbl_nbio} presents results from my non-blocking I/O implementation. I started $N$ requests
simultaneously.

\begin{table}[H]
  \centering
  \begin{tabular}{lr}
    & {\bf Time (s)} \\
    \hline
    Run 1 & 0 \\
    Run 2 & 0 \\
    Run 3 & 0 \\
    Run 4 & 0 \\
    Run 5 & 0 \\
    Run 6 & 0 \\
    \hline
    Average & 0 \\
  \end{tabular}
  \caption{\label{tbl_nbio}Non-blocking I/O executions terminate in a mean of $i$ seconds.}
\end{table}

\paragraph{Discussion.} Surprisingly, the sequential execution ran fastest. I'm
not sure why.

\section*{Part 3: Amdahl's Law and Gustafson's Law}
I did XXX to measure the sequential portion of {\tt paster\_parallel}. Over 3 runs,
it took an average of M seconds. Amdahl's Law\ldots

\end{document}
